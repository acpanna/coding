\documentclass[a4paper]{article}
\usepackage{ngerman}
\usepackage{amssymb}
\author{Cornelius Schwarz}
\title{Die Jordansche Normalform}
\date{22. Januar 2004}
\begin{document}
\maketitle
\begin{section}{Aufgabenstellung von Normalformen}
Um zu entscheiden, ob zwei Matrixen zueinander "ahnlich sind, also den selben Endomorphismus $f$ repr"asentieren, transformiert man sie ein eine geeignete Normalform und vergleicht diese. Die Transitivit"at der "Ahnlichkeitsrelation bedeutet ja gerade: $A \sim N, N \sim B \Leftrightarrow A \sim B$.\\
Diese Aufgabe erf"ullen alle Normalformen. Folgendes Beispiel soll motivieren, warum es von Vorteil sein kann, eine Normalform von besonders einfacher Gestalt zu haben.
\begin{subsection}{Beispiel}
Es soll die Potenz von $A^k$ berechnet werden, dabei sei $A \in \mathbb{K}^{n \times n}, k \in \mathbb{N}$. An Stelle $A^k$ direkt zu berechnen, was mit einem nativen Algorithmus $k$ Matrixmultiplikationen mit $O(n^3)$ Operationen bedeuten w"urde, kann man auch $A$ in eine Normalform $N$ transformieren und dann $N^k$ berechnen. Es sei $A=P^{-1}nP$, dann gilt:
$$A^k=(P^{-1}NP)^k = P^{-1} N P P^{-1} N P \ldots P^{-1} N P = P^{-1} N^k P$$
Die Anzahl der Operationen, um $P$ zu berechnen h"angt nur von der Dimension der Matrix ab, ist also unabh"angig von $k$. Wenn $k >> n$ ist und $N$ einfache Gestalt, z.B. Diagonalmatrix, hat, lohnt sich die Transformation.
\end{subsection}
\end{section}
\begin{section}{Elementarteiler, die rationale Normalform}
Um einen Endomorphismus in einer Normalform anzugeben, brauchen wir Merkmale, die nicht von der gew"ahlten Basis abh"angen.
Ein Beispiel daf"ur ist das charakteristische Polynom. Allerdings bestimmt dieses den Endomorphismus nicht eindeutig. Die folgenden beiden Matrix haben das gleiche charakteristische Polynom $(X-1)^5$, sind aber nicht zueinander "ahnlich:
$$\left (\begin{array}{ccccc}
1 & 0 & 0 & 0 & 0\\
0 & 1 & 0 & 0 & 0\\
0 & 0 & 1 & 0 & 0\\
0 & 0 & 0 & 1 & 0\\
0 & 0 & 0 & 0 & 1
\end{array}\right )
\qquad \left ( \begin{array}{ccccc}
1 & 0 & 0 & 0 & 0\\
1 & 1 & 0 & 0 & 0\\
0 & 1 & 1 & 0 & 0\\
0 & 0 & 0 & 1 & 0\\
0 & 0 & 0 & 0 & 1\\
\end{array} \right )
$$
Ein Vergleich der beiden Matrixen liefert, da"s die erste Matrix als Minimalpolynom $(X-1)$ und die zweite $(X-1)^3$ hat.\\
Allerdings reicht die Vorraussetzung ''Gleiches Minimalpolynom'' auch nicht aus:
$$\left ( \begin{array}{cccccc}
1 & 0 & 0 & 0 & 0\\
1 & 1 & 0 & 0 & 0\\
0 & 1 & 1 & 0 & 0\\
0 & 0 & 0 & 1 & 0\\
0 & 0 & 0 & 1 & 1
\end{array} \right )$$
Wir erweitern also noch einmal und kommen zu dem Beriff des
\begin{subsection}{Satz + Definition: Elementarteiler}
Sei $M$ ein endlich erzeugter $R$-Modul "uber einem euklidischen Ring $R$ mit $Ann(M)\not=(0)$, dann gibt es $m_1,\ldots,m_s \in  M$ mit $$M=\bigoplus_{j=1}^s Rm_i$$ und die Erzeugenden $k_j$ der Ordnungsideale $Ord(m_j)\not=(0)$ gen"ugen den Teilbarkeitsbedingungen $k_{j+1}|k_j, j=1,\ldots,s-1$. Diese $k_i$ sind bis auf Assoziiertheit eindeutig bestimmt und hei"sen die \emph{Elementarteiler} von $M$ und bestimmen $M$ bis auf $R$-Isomorphie.
\label{elteiler}(vgl. Satz 3.9.4 \cite{vorlesung})\\\\
\end{subsection}
Das sieht auf den ersten Blick ziemlich abstrakt aus. Wir betrachten allerdings nur den Fall $R=\mathbb{K}[X]:$ Hier ist $M = V_f$. $Ann(M)$ wird erzeugt von dem Minimalpolynom $m_1$. Zu dem Minimalpolynom gibt es eine Vektor $v$ der dieses als Ordnunspolynom hat(vgl. \cite{min}). Wir betrachten nun den Unterraum $\mathbb{K}[X] v$. Eine $\mathbb{K}$ Basis dieses Unterraums $U_1$ von $V$ bekommt man, indem man mittels Gau"salgorithmus sukzessiv $v,f(v), f^2(v), \ldots$ zur Basis dazunimmt, bis diese linear abh"angig werden(vgl. \cite{gauss}).\\
Dann betrachten wir $V/U_1$. $Ann(V/U_1)$ in $\mathbb{K}[X]$ wird ebenfalls von einem eindeutigen normierten Polynom $m_2$ erzeugt zu dem es wieder einen Vektor $v_2$ gibt, der dieses als Ordnungspolynom hat. Eine $\mathbb{K}$ Basis von $U_2$ erh"alt man analog zu oben. Es gilt: $Ann(V) \subset Ann(V/U_1)$, da mit $p*v=0$ auch $p*v+U=(p*v)+U=0+U=0$ ist $\Leftrightarrow m_2 | m_1$. Wir betrachten nun $(V/U_1)/U_2$, also $V/<U_1 + U_2>$ und iterieren das Verfahren bis $V/<U_1 + \ldots + U_{s+1}>=\{0\}$ ist. Aufgrund des Eindeutigkeitssatzes \ref{elteiler} haben wir damit die Elementarteiler berechnet.
\end{section}
\begin{section}{Prim"arkomponente, die jordansche Normalform}
\begin{subsection}{Satz + Definition: Prim"arkomponente}
Sei $R$ ein euklidischer Ring, $M$ ein Torsionsmodul, also $Ord(m) \not= 0\ \forall  \ m\in M$,$P$ ein maximales Ideal, da $R$ HIR, $\exists p \in \ R \ prim: P=(p)$.
Sei $\Phi$ die Menge der maximalen Ideale von $R$
$$M_P := \{ m | m \in M, \exists i \in \mathbb{N}: Ord(m) = (p^i)\}$$
Es gilt:
\begin{enumerate}
\item $$M_P \textnormal{ ist Untermodul f"ur alle } P.$$
\item $$M = \bigoplus_{P \in \Phi} M_P$$
\item $$Ann(M) \subset P \Longleftrightarrow M_P \not= \{0\}$$ \label{teilt}
\end{enumerate}
Die $M_P$ hei"sen \emph{Prim"arkomponeten} von $M$. (vgl. Satz 3.10.10 \cite{vorlesung})\\\\
\end{subsection}
Wiederum betrachten wir den Fall $\mathbb{K}[X]$. Desweiteren setzten wir vorraus, da"s das Minimalpolynom in Linearfaktoren zerf"allt. Dies kann man immer erreichen, ggf. nach "Ubergang zum Zerf"allungsk"orper. Die Prim"arkomponenten sehen also folgenderma"sen aus: $\{ v \in V, \exists i \in \mathbb{N}: Ord(v) = (X-\lambda)^i \Leftrightarrow (A-\lambda)^i v = 0 \}$. Falls $M_P \not= \{0\}$, so gilt doch $Ord(v) | Ann(V)$, also $X-\lambda$ ein Teiler des Minimalpolynoms. Wir brauchen also nur $(A-\lambda)^r = 0 $ zu fordern, wobei $r$ die Potenz des Primfaktors $X-\lambda$ im Minimalpolynom ist. Aufgrund von \ref{teilt} k"onnen wir uns auf $p=X-\lambda$ mit $\lambda$ Nullstelle des Minimalpolynoms also Eigenwert beschr"anken ($p | m$). Trivialerweise ist $A$ auf $M_P$ invariant. Wir schr"anken die Matrix also auf Prim"arkomponente ein. Das Minimalpolynom auf $M_P$ lautet dann, $(X-\lambda)^r$, wobei $r$ der Index in dem Minimalpolynom auf $V$ ist. Nach \ref{elteiler} erhalten wir eine Zerlegung in ein direkte Summe $\mathbb{K}[X]$ zyklische Unterr"aume. Mit $M_{P_\lambda}$ gleich der Prim"arkomponente zum Eigenwert $\lambda$ gilt also:
\begin{equation} M_{P_\lambda} = \bigoplus_{i=1}^s \mathbb{K}[X] m_i \label{sum} \end{equation}
Die Elementarteiler als Teiler des Minimalpolynoms haben die Gestalt $(X-\lambda)^{\delta_i}$ mit $r=\delta_1 \leq .. \leq \delta_s$. Jetzt betrachten wir den Untermodul $U_{\delta_i}:=\mathbb{K}[X] (X-\lambda)^{\delta_i}$ mit zugeh"origem Vektor $v$. Eine $\mathbb{K}$ Basis bilden die Vektoren $v, (A-\lambda) v, \ldots, (A-\lambda)^{\delta_i-1} v$. $A$ hat bez"uglich dieser Basis Jordanblockgestalt auf $U_{\delta_i}$:
$$\left ( \begin{array}{cccc}
\lambda &         &        & \\
1       & \lambda &        & \\
        & \ddots  & \ddots & \\
        &         & 1      & \lambda
\end{array} \right )$$
\begin{subsection}{Beispiel}
\label{bsp-1}
Sei  $$A := \left ( \begin{array}{ccc}
0 & 0 & 1 \\
1 & 0 & 2 \\
0 & 1 & 3 
\end{array} \right )$$
Diese Matrix ist in Rationaler Normalform. Das Minimalpolynom l"asst sich also direkt ablesen: $x^3 + 3 x^2 + 2 x +1 = (x + 4)^3$. Wir betrachten die einzige Prim"arkomponente $P_1$ zum Eigenwert 1.
$$(A-1)=\left ( \begin{array}{ccc}
4 & 0 & 1\\
1 & 4 & 2\\
0 & 1 & 2
\end{array} \right )
\quad (A-1*E)^2=\left ( \begin{array}{ccc}
1 & 1 & 1\\
3 & 3 & 3\\
1 & 1 & 1
\end{array} \right )
\quad (A-1*E)^3=\left ( \begin{array}{ccc}
0 & 0 & 0\\
0 & 0 & 0\\
0 & 0 & 0
\end{array}\right )
$$
Eine Basis der Prim"arkomponente liefern uns also die Einheitsvektoren. Der Vektor $e_1 = (1,0,0)^T$ hat als Ordnungspolynom $(X+4)^3$. Es gibt also keine weiteren Elementarteiler. Die Jordanbasis bekommen wir also mit $e_1, (A-1) * e_1 = (4, 1, 0)^T, (A-1)^2 * e_1 = (1, 3, 1)^T$. Die Jordanform sieht bez"uglich dieser Basis dann
$$\left (\begin{array}{ccc}
1 & 0 & 0\\
1 & 1 & 0\\
0 & 1 & 1\\
\end{array} \right )$$
aus.
\end{subsection}
\end{section}
\begin{section}{Algorithmus JNF Variante 1: Vorlesung bzw. L"uneburg}
\begin{subsection}{Der Algorithmus}
Sei $\mathbb{K}$ K"orper, $\mathbb{V} \mathbb{K}-$Vektorraum mit $n := \dim(\mathbb{V})$ und sei $\varphi \in End_\mathbb{k}(\mathbb{V})$ und $A$ seine darstellende Matrix bez"uglich der kanonischen Einheitsvektoren in Quelle und Ziel. \\
\begin{description}
\item [Initialisierung] Berechne die Eigenwerte von $A$.
\item [Iterationschritt] F"ur jeden Eigenwert $\lambda_i$
 \begin{enumerate}
  \item Berechne eine Basis der Prim"arkomponete $(V_\varphi)_{(X-\lambda_i)}$
  \item Schr"anke $\varphi$ ein auf $V_i$ und berechne die Elementarteiler $(m_{i,0},\ldots,m_{i,s_i})$ und zugeh"orige Vektoren $v_{i,j}$ mit $Ord(v_{i,j})=(m_{i,j})$
 \end{enumerate}
\end{description}
$m_{i,j}=(X-\lambda_i)^{\delta_{i,j}}$, wobei $\delta_{i,0}=r_i \geq \delta_{i,1} \geq \ldots \geq \delta_{i,s_i}$. Die Jordanmatrix hat also zum Eigenwert $\lambda_i$ genau $s_i$ Jordanbl"ocke, die jeweils die Gr"ose $\delta_{i,j}$ haben. Eine Jordanbasis ergibt sich aus dem Hintereinaderreihen von $(v_{i,j}, (A-\lambda_i)*v_{i,j}, \ldots, (A-\lambda_i)^{\delta_{i,j-1}}*v)$.
\end{subsection}
\end{section}
\begin{section}{Algorithmus JNF Variante 2: Repetitorium der Linearen Algebra}
\begin{subsubsection}{Lemma}
Sei $f \in End(V,V)$, dann gilt:
$$Kern(f^i) \subseteq Kern(f^{i+1}), i \in \mathbb{N}$$
Beweis: trivial \label{existens}
\end{subsubsection}
\begin{subsubsection}{Definition: Stufenbasis}
$S$ hei"st Stufenbasis zur Prim"arkomponente $P_i$, falls gilt:
\begin{enumerate}
\item $S=S_1 \cup \ldots S_k, S_i \cap S_j = \emptyset, i \not= j$
\item $S_1 \cup \ldots \cup S_j $ ist Basis von $Kern((A-\lambda_i)^j)$, $j=1,\ldots,k$
\end{enumerate}
\end{subsubsection}
Zur Veranschaulichung:
$$
\begin{array}{|cccc||c|}
\hline
v_8 & v_9 &     &     & S_3\\\hline
v_5 & v_6 & v_7 &     & S_2\\\hline
v_1 & v_2 & v_3 & v_4 & S_1\\\hline
\end{array}
$$
$S_1$ ist Basis von $Kern(A-\lambda_i)$, $S_2 \cup S_!$ ist Basis von $Kern(A-\lambda_i)^2$. $S_2$ enth"alt also die Basisvektoren, die f"ur eine Basis von $Kern(A-\lambda_i)^2$ noch ''fehlen''.
\begin{subsection}{Der Algorithmus}
\begin{description}
\item[Eingabe] Eine Matrix "uber einen $\mathbb{K}-Vektorraum \mathbb{V}$, sowie deren Eingenwerte.
\item[Iteration] Durchlaufe die Eigenwerte von $A$ mit $\lambda$
\begin{enumerate}
\item Berechne eine Stufenbasis der Prim"arkomponente $V_\varphi)_{(X-\lambda)}$
\item Berechne aus der Stufenbasis eine Jordanbasis
\end{enumerate}
\end{description}
\textbf{Zur Stufenbasis}\\
Die Existens folgt \ref{existens}. Die Berechnung folgt mittels Gauss. Man nutzt an dieser Stelle aus, da"s bei der Berechnung von $Kern(X-\lambda_i)^j$ die Nullzeilen von der Berechnung von $Kern(X-\lambda_i)^{j-1}$ erhalten bleiben. Man mu"s bei der Berechnung also nur die neuen Nullzeilen als freie Parameter betrachten.\\
\textbf{Zur Jordanbasis}\\
Hier nutzt man die Tatsache, da"s $Ord(v) = (X-\lambda_i)^j \Rightarrow Ord((X-\lambda_i) v)= (X-\lambda_i)^{j-1}$ gilt. Denn wenn gilt: $(A-\lambda_i)^j v=0 \Leftrightarrow (A-\lambda_i)^{j-1}((A-\lambda_i)v)=0$. Also hat $(A-\lambda_i)v$ als Ordnungspolynom $(X-\lambda_i)^{j-1}$ liegt also im $Kern(A-\lambda_i)^{j-1}$ also ''eine Stufe tiefer''. Ausserdem nutzen wir aus, dass f"ur lineare unabh"nagige $v_i$ mit $Ord(v_i)=(X-\lambda_i)^j, j > 1$ gilt: $(X-\lambda_i)v_i$ sind linear unabh"angig, denn
$$ \sum_{i=1}^k \kappa_i (A-\lambda_i) v_i = 0 \Rightarrow (A-\lambda_i) \sum_{i=1}^k \kappa_i v_i = 0 $$
Also liegt die Linearkombination der $v_i$ im $Kern(A-\lambda_i)$. Da die $v_i$ als linear unabh"angig vorrausgesetzt waren, gilt dies nur, falls $j=1$ ist. \\
Folgender Algorithmus liefert damit eine Jordanbasis:
\begin{description}
\item[Initialisierung] $T_k:=R_k:=S_k$
\item[Schritt j] 
\begin{itemize}
\item Basis von $Kern(A-\lambda_i)^{k-j}: S_1 \cup \ldots S_{k-j} \cup T_{k-j+1}$ 
\item ersetzte $S_{k-j}$ durch $X:=\{(A-\lambda_i) * v | v \in T_{k-j+1}\}$
\item erweitere diese Menge durch $R_{k-j}$ zu einer Basis von $Kern(A-\lambda_i)^{k-j}$
\item setze $T_{k-j}:=X \cup R_{k-j}$ 
\end{itemize}
\end{description}
Zur Veranschaulichung: \\
Betrachten wir noch einmal obige Skizze, dann starten wir mit $v_8, v_9$ in der Stufe $S_3$. Beim "Ubergang zu Stufe $S_2$ berechnen wir als erstes $(A-\lambda_i) v_8, (A-\lambda_i) v_9$ und nehmen dann noch einen zu diesen Vektoren  linear unabh"angigen Vektor  aus der Menge $\{v_5, v_6, v_7\}$ hinzu. Sei dies o.E. $v_7$, dann ist $v_1, \ldots, v_4, (A-\lambda_i) v_8, (A-\lambda_i) v_9,v_7$ eine Basis von $(A-\lambda_i)^2$. Wir iterieren und betrachten $(A-\lambda_i)^2 v_8, (A-\lambda_i)^1 v_9, (A-\lambda_i) v_7$ und erweitern wieder zu einer Basis, z.B. durch $v_4$.
$$\begin{array}{|cccc||c|}\hline
v_8                 & v_9                 &                   &     & S_3 \\\hline
(A-\lambda_i) v_8   & (A-\lambda_i) v_9   & v_7               &     & S_2 \\\hline
(A-\lambda_i)^2 v_8 & (A-\lambda_i)^2 v_9 & (A-\lambda_i) v_7 & v_4 & S_1 \\\hline 
\end{array}
$$
Kontruktion der Jordanbasis:\\
Zu $r \in R_j$ existiert ein $j \times j \ Jordanblock$ mit der Basis: $r,\ldots, (A-\lambda)^j*r$. Durch Hintereinanderschreiben dieser Basen erh"alt man die Jordanbasis von $A$. \\
Im Beispiel ist die Jordanbasis:
$$v_8, (A-\lambda_i)v_8, (A-\lambda_i)^2v_8, v_9, (A-\lambda_i)v_9, (A-\lambda_i)^2v_9, v_7, (A-\lambda_i) v_7, v_4$$
Wir bekommen demnach $2$ $3\times3$ Jordanbl"ocke, $1$ $2\times2$ Jordanblock und $1$ $1\times1$ Jordanblock. 
\end{subsection}
\begin{subsection}{Beispiel}
Betrachten wir wieder die Matrix:
$$A:=\left ( \begin{array}{ccc}
0 & 0 & 1 \\
1 & 0 & 2 \\
0 & 1 & 3
\end{array}\right )$$
$$(A-1 * E):= \left ( \begin{array}{ccc}
4 & 0 & 1\\
1 & 4 & 2 \\
0 & 1 & 2
\end{array} \right ) \ \begin{array}{c}\mbox{\tiny Zeilenstufenform} \\ \Longrightarrow \end{array}
\left ( \begin{array}{ccc}
4 & 0 & 1 \\
0 & 1 & 2 \\
0 & 0 & 0
\end{array} \right )$$
Eine Basis von $Kern(A-1*E)$ liefert demnach der Vektor $v_1:=(1,3,1)^T, S_1:=\{v_1\}$
$$(A-1*E)^2 \left ( \begin{array}{ccc}
1 & 1 & 1 \\
3 & 3 & 3 \\
1 & 1 & 1
\end{array} \right ) \ \begin{array}{c}\mbox{\tiny Zeilenstufenform} \\ \Longrightarrow \end{array}
\left ( \begin{array}{ccc}
1 & 1 & 1 \\
0 & 0 & 0 \\
0 & 0 & 0
\end{array} \right )$$
Als neue Nullzeile bekommen wir die zweite Zeile. Demnach bekommen wir den Vektor $v_2:=(4,1,0)^T, S_2=\{v_2\}$\\
Die dritte Potenz von $(A-1*E)$ war die Nullmatrix und liefert uns mit der ersten Zeile als zus"atzliche Nullzeile den ersten Einheitsvektor. $v_3 = (1,0,0)^T, S_3=\{v_3\}$. Wir bekommen hier mit $v_3, (A-1*E) v_3, (A-1*E)^1 v_3$ die selbe Basis wie zuvor (\ref{bsp-1}).
\end{subsection}
\end{section}
\begin{thebibliography}{99}
\bibitem[lina02]{vorlesung} Prof. Dr. A. Kerber. Lineare Algebra, WS 2002/2003
\bibitem[rep]{rep} Dr. Michael Holz, Dr. Detlef Wille, Repetitorium der Linearen Algebra  Teil 2, Binomi Verlag
\bibitem[hl]{lb} H. L"uneburg, Vorlesung "uber Lineare Algebra, BI Wissenschaftsverlag, 1993
\bibitem[hb]{min} H. Buchholzer, Vortrag "uber Minimalpolom am 08.01.2004
\bibitem[bvkp]{gauss} Binca Valentin, Katja P"ollman, Vortrag "uber Gau"s Algorithmus nach L"uneburg
\end{thebibliography}
\end{document}
