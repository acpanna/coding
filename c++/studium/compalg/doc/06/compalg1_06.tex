\documentclass{article}
\usepackage[latin1] {inputenc}
\usepackage{amsmath, amsfonts ,amssymb, ngerman}
\pagestyle{myheadings}

\markright{Seminar Computeralgebra 1, Thema 6, Tobias Kreisel}

\newtheorem{Sat}{Satz}

\newtheorem{Bem}{Bemerkung}

\newtheorem{Fol}{Folgerung}

\newcommand{\ggt}[2]{\ensuremath{\mathrm{ggT}(#1,#2)}}
\newcommand{\Kern}[1]{\ensuremath{\mathrm{Kern}(#1)}}

\begin{document}

%%%%%%%%%%%%%%%%%%%%%%%%%%%%%%%%%%%%%%%%%%%%%%%%%%%%%%%%%%%%%%%%%%

\section*{Berechnung ganzzahliger Determinanten}

\begin{Sat}[Der chinesische Restsatz]
Seien $R$ ein euklidischer Bereich, \\
$m_0,\dots,m_{r-1}$ paarweise teilerfremd (also insbesondere $\ggt{m_i}{m_j}=1 \\
\forall i \ne j$). Definiere Ringhomomorphismus:
$$ \pi_i : R \longrightarrow R / (m_i), \quad f \longmapsto f\bmod m_i .$$
Kombiniert f"ur alle i ergibt dies:
\begin{equation*}
	\begin{split}
		\chi = \pi_0 \times \dots \times \pi_{r-1} : R& \longrightarrow R / (m_0) \times \dots \times R / 	(m_{r-1}),\\
		f& \longmapsto (f \bmod m_0, \dots , f \bmod m_{r-1}).
	\end{split}
\end{equation*}
Behauptung: $\chi$ ist surjektiv mit $\Kern{\chi}=(m)$, wobei $m := m_0 \cdot \dotso \cdot m_{r-1}$.
\end{Sat}
Beweis:
\begin{description}
	\item[\textmd{Kern:}] Sei $f \in \Kern{\chi}$ \\
		$\Longleftrightarrow \chi(f) = (f \bmod m_0,\dots,f \bmod m_{r-1}) = (0,\dots,0)$ \\
		$\Longleftrightarrow m_i \mid f \; \forall i: 0 \le i < r$ \\
		$\Longleftrightarrow m \mid f \Rightarrow \Kern{\chi}=(m).$
	\item[\textmd{Surjektivit"at:}]
	Zeige: Es existiert $l_i \in R$ mit $\chi(l_i)=e_i \; (0 \le i < r)$, wobei $e_i=(0,\dots,0, 1, 0,\dots, 0) \in R/(m_0) \times \dots \times R/(m_{r-1})$. \\
	Dies gen"ugt, da: Sei $v=(v_0 \bmod m_0,\dots,v_{r-1} \bmod m_{r-1}) \in R/(m_0) \times \dots \times R/(m_{r-1})$ beliebig, $v_0, \dots, v_{r-1} \in R$. Dann gilt:
	\begin{equation*}
		\begin{split}
			\chi(\sum_{i=0}^{r-1} v_i l_i) \stackrel{\chi \; Hom.}{=}& \sum_{i=0}^{r-1} \chi(v_i) \chi(l_i) = \sum_{i=0}^{r-1} (v_i \bmod m_0, \dots, v_i \bmod m_{r-1}) \cdot e_i \\
			=& \sum_{i=0}^{r-1} (0, \dots, 0, v_i \bmod m_i, 0, \dots, 0) = v
		\end{split}
	\end{equation*}
	\underline{oBdA $i=0$:}\\
	Der erweiterte euklidische Algorithmus (EEA) angewandt auf $m_1 \cdot \dotso \cdot m_{r-1} = \frac{m}{m_0}$
	und $m_0$ liefert $s, t \in R$ mit $s \cdot \frac{m}{m_0} + t \cdot m_0 = 1$ (m"oglich, da $m_i$ teilerfremd). \\
	Setze also $l_0 = s \cdot \frac{m}{m_0}$, dieses hat die gew"unschte Eigenschaft:
	\begin{equation*}
		\begin{split}
			l_0 = s \cdot \frac{m}{m_0} \equiv s \cdot \frac{m}{m_0} + t \cdot m_0 \equiv& 1 \pmod {m_0}\\
			\equiv& 0 \pmod {m_j} \; (j=1,\dots,r-1).
		\end{split}
	\end{equation*}
	Somit folgt: $\chi(l_0) = e_0$.
	\begin{flushright}
		$\square$
	\end{flushright}
\end{description}

\begin{Fol}
Da wir $\Kern{\chi}$ kennen, k"onnen wir folgenden Ringisomorphismus definieren:
\begin{equation*}
	\begin{split}
		\tilde{\chi} : R/(m)& \longrightarrow R/(m_0) \times \dots \times R/(m_{r-1}), \\
		f \bmod m& \longmapsto (f \bmod m_0, \dots , f \bmod m_{r-1}).
	\end{split}
\end{equation*}
\end{Fol}

\subsection*{Algorithmus chinese1}
\begin{description}
		\item[\underline{input:}] $m_0, \dots, m_{r-1} \in R$ paarw. teilerfremd; $v_0, \dots, v_{r-1} \in R$.
		\item[\underline{output:}] $f \in R$ mit $f \equiv v_i \pmod m_i \quad (0 \le i < r)$
		\item[Schritt 1:] $m \longleftarrow m_0 \cdot \dotso \cdot m_{r-1}$
		\item[Schritt 2:] Berechne $s_i, t_i \in R$ mit $$s_i \cdot \frac{m}{m_i} + t_i \cdot m_i = 1 \quad \mbox{(EEA)}$$ $c_i \longleftarrow v_i \cdot s_i \bmod m_i$
		\item[Schritt 3:] $\text{Ausgabe von} \; \sum_{i=0}^{r-1} c_i \cdot \frac{m}{m_i}$
\end{description}
Da obenstehender Algorithmus in jedem Schritt auf die explizite Berechnung von $m := m_0 \cdot \dotso \cdot m_{r-1}$ angewiesen ist, erzeugt man unn"otig gro\ss e Werte. Deshalb konstruieren wir noch folgenden Algorithmus:
\subsection*{Algorithmus chinese2}
\begin{description}
	\item[\underline{input/output:}] siehe chinese1
	\item[Schritt 0:] Setze $f_0 := v_0 \qquad n_0 := m_0$
	\item[Schritt i:] F"ur $i=1, \dots, r-1$ berechne $a \cdot n_{i-1} + b \cdot m_i = 1.$ Setze: \\
		$z := (v_i - f_{i-1}) \cdot a \pmod {m_i} \\
		f_i := f_{i-1} + z \cdot n_{i-1} \\
		n_i := n_{i-1} \cdot m_i$
\end{description}
Beweis (per Induktion):
\begin{description}
	\item[\underline{$i=0:$}] $f_0 \equiv v_0 \bmod m_0 \qquad \surd$
	\item[\underline{$i\!-\!1 \rightarrow i:$}] Zeige \\
		a) f"ur $j=i: \; f_i \equiv v_i \pmod {m_i}.$ \\
		b) f"ur $j<i: \; f_i \equiv v_j \pmod {m_j}.$
	\begin{equation*}
		\begin{split}
			\text{zu a) Es gilt:} \quad & a \cdot n_{i-1} + b \cdot m_i = 1 \Rightarrow a \cdot n_{i-1} \equiv 1 \pmod {m_i} \\
						& \Longrightarrow z \cdot n_{i-1} = (v_i - f_{i-1}) \cdot \underbrace{a \cdot n_{i-1}}_{\equiv 1 \pmod {m_i}} \\
						& \Longrightarrow f_i = f_{i-1} + \underbrace{z \cdot n_{i-1}}_{\equiv (v_i - f_{i-1}) \pmod {m_i}} \equiv v_i \pmod {m_i} \\
			\text{zu b) Es gilt:} \quad & n_{i-1} = m_0 \cdot \dotso \cdot m_{i-1} \equiv 0 \pmod {m_j} \\
						& \Longrightarrow f_i = f_{i-1} + \underbrace{z \cdot n_{i-1}}_{\equiv 0 \pmod {m_j}} \equiv f_{i-1} \equiv v_j \pmod {m_j} \\
		\end{split}
	\end{equation*}
	\begin{flushright}
		$\square$
	\end{flushright}
\end{description}

\subsection*{Algorithmus Modulare Determinatenberechnung}
\begin{description}
	\item[\underline{input:}] $A \in \mathbb{Z}^{n \times n}$, mit $|a_{ij}| \le B \quad \forall i,j.$
	\item[\underline{output:}] $\det(A) \in \mathbb{Z}.$
	\item[Schritt 1:] $C \longleftarrow n^{\frac{n}{2}} \cdot B^n$ (Hadamard Absch"atzung von $\det(A)$ nach oben) \\
		W"ahle $r$ Primzahlen $m_i$, sodass: $\prod_{i=0}^{r-1} m_i > C$. Hierdurch "'vergr"o"sert"' man gewisserma"sen
		den eingangs erw"ahnten $\Kern{\chi}$ und stellt somit sicher, dass $\chi$ Isomorphismus ist
		(f"ur den gesuchten Wert von $\det(A)$).
	\item[Schritt 2:] Berechne $\overline{A} \equiv A \pmod {m_i}$ f"ur $i=0, \dots, r-1.$ \\
		Dies lie"se sich sehr bequem parallelisieren.
	\item[Schritt 3:] Berechne $v_i \in \lbrace 0, \dots, m_{i-1} \rbrace$, sodass $v_i \equiv \det(A) \pmod {m_i}$
	\item[Schritt 4:] Berechne $d \equiv v_i \pmod {m_i}$ mit Hilfe von chinese1/2. Gebe $d$ aus.
\end{description}

\end{document}
