%% LyX 1.3 created this file.  For more info, see http://www.lyx.org/.
%% Do not edit unless you really know what you are doing.
\documentclass[english,german,german , dvips]{slides}
\usepackage[T1]{fontenc}
\usepackage[latin1]{inputenc}
\pagestyle{plain}
\usepackage{amsmath}
\usepackage{color}
\usepackage{amssymb}

\makeatletter

%%%%%%%%%%%%%%%%%%%%%%%%%%%%%% LyX specific LaTeX commands.
\newcommand{\lyxline}[1]{
  {#1 \vspace{1ex} \hrule width \columnwidth \vspace{1ex}}
}
\newcommand{\noun}[1]{\textsc{#1}}

%%%%%%%%%%%%%%%%%%%%%%%%%%%%%% Textclass specific LaTeX commands.
 \newcounter{slidetype}
 \setcounter{slidetype}{0}
 \newif\ifLyXsNoCenter
 \LyXsNoCenterfalse
 \newcommand{\noslidecentering}{
    \LyXsNoCentertrue%
 }
 \newcommand{\slidecentering}{
    \LyXsNoCenterfalse%
 }
 \newcommand{\lyxendslide}[1]{
    \ifLyXsNoCenter%
         \vfill%
    \fi%
    \ifcase \value{slidetype}%
         \or % no action for 0
         \end{slide} \or%
         \end{overlay} \or%
         \end{note}%
    \fi%
    \setcounter{slidetype}{0}
\visible
 }
 \AtEndDocument{\lyxendslide{.}}
 \newcommand{\lyxnewslide}[1]{
    \lyxendslide{.}
    \setcounter{slidetype}{1}
    \begin{slide}
 }

%%%%%%%%%%%%%%%%%%%%%%%%%%%%%% User specified LaTeX commands.
\usepackage[T1]{fontenc}
\usepackage[latin1]{inputenc}
\usepackage{geometry}
\geometry{verbose,paperwidth=310mm,paperheight=210mm,tmargin=10mm,bmargin=10mm,lmargin=15mm,rmargin=15mm,headheight=0mm,headsep=0mm,footskip=0mm}
\setlength\parskip{\smallskipamount}
\setlength\parindent{0pt}
\pagestyle{plain}
\usepackage{amsmath}
\usepackage{amssymb}
\usepackage{array}
\usepackage{amsmath}
%\usepackage[dvips]{color}
%\usepackage[dvips]{graphicx}
\usepackage{amssymb}

\makeatletter

\usepackage{amsmath, amsfonts ,amssymb, german, theorem} 
\theorembodyfont{\normalfont}
\theoremstyle{break}
\newtheorem{Def}{\underline{Definition}}

\theorembodyfont{\normalfont}
\theoremstyle{break}
\newtheorem{Sat}{\underline{Satz}}

\theorembodyfont{\normalfont}
\theoremstyle{break}
\newtheorem{Bem}{\underline{Bemerkung}}

\theorembodyfont{\normalfont}
\theoremstyle{break}
\newtheorem{Bei}{\underline{Beispiel}}

\newcommand{\ggT}{ \textrm{ggT} }
\newcommand{\kgV}{ \textrm{kgV} }
\newcommand{\R}{\textrm{r} }

\usepackage{babel}
\makeatother

\usepackage{babel}
\makeatother
\begin{document}
\begin{flushleft}\textbf{\large Seminar: Computeralgebra 1}\\
\end{flushleft}

\lyxline{\Huge}\vspace{-1\parskip}
\begin{center}\textcolor{red}{\noun{\Huge Thema 7 : }}\\
\textcolor{red}{\noun{\Huge }}\\
\textcolor{red}{\noun{\Huge Berechnung des Minimalplolynoms zu einem
Endomorphismus }}\\
\textcolor{red}{\noun{\Huge }}\\
\textcolor{red}{\noun{\Huge Die Algorithmen Ordpoly und Minpoly}}\end{center}{\Huge \par}
\lyxline{\Huge}

\begin{flushleft}Hannes Buchholzer \end{flushleft}


\lyxnewslide{}

\noindent \textbf{\textcolor{black}{\underbar{\Huge Generalvoraussetzungen:}}}\\
\\
 \textcolor{black}{\huge Seien $K$ ein K�rper, $R$ ein euklidischer
Ring, $V$ ein $K$-Vektorraum mit $dimV=n\:(n\in\mathbb{N})$, $\tau\in End(V)$
, und $M$ ein eindlich erzeugter Modul �ber $R$.}{\huge \par}


\lyxnewslide{}

\textcolor{blue}{\Huge Gliederung:}\textcolor{blue}{\huge }\\
{\huge \par}

\textcolor{black}{\huge 1. Wiederholung}\\
{\huge \par}

\textcolor{black}{\huge 2. Der Algorithmus Ordpoly}\\
{\huge \par}

\textcolor{black}{\huge 3. Der Algorithmus Minpoly}{\huge \par}

\vspace*{\fill}


\lyxnewslide{}

\textcolor{blue}{\Huge 1 Wiederholung}{\Huge \par}

\begin{enumerate}
\item Der Polynomring $K[X]$ ist ein euklidischer Ring, insbesondere ein
Hauptidealring (HIR). 
\item Das Annulatorideal von M ist $A(M):=\{ r\in R\mid rm=0\:\forall m\in M\}$. 
\item Das Ordnungsideal von $m\in M$ ist $O(m):=\{ r\in R\mid r\cdot m=0\}$.
$O(m)$ ist ein Ideal. 
\item Ein Element $m\in M$ hei�t Torsionselement, wenn $O(m)\supsetneqq\{0\}$.
Ist jedes $m\in M$ Torsionselement, so hei�t $M$ Torsionsmodul.
\vspace*{\fill}
\end{enumerate}

\lyxnewslide{}

\textcolor{blue}{\huge 1 Wiederholung}\\
{\huge \par}

\begin{Def}[ {\(K[X]\)-Modul \(V_{\tau}\)} ]
Sei \( f \in K[X]\) und \(v \in V\). Durch die Definition 
\( f \cdot v :=  (f(\tau))(v) \) wird \(V\) zu einem \(K[X]\)-Modul, bezeichnet mit \(V_{\tau}\).\\*[16pt]

\end{Def}









\begin{Bem}

1. F�r Elemente $k\in K\subset K[X]$ folgt aus diese Definition:
$k\cdot v=k\cdot\tau^{0}(v)=$ \\
$k\star v$ , wobei $\star$ die Skalarmultiplikation im $K$-Vektorraum
$V$ bezeichnet. \\
2. Der Modul $V_{\tau}$ ist endlich erzeugt. \\
3. Au�erdem ist $V_{\tau}$ ein Torsionsmodul.  \\
 \end{Bem}

\vspace*{\fill}


\lyxnewslide{}

\textcolor{blue}{\Huge 2. Der Algorithmus Ordpoly}{\Huge \par}

\selectlanguage{english}
\begin{itemize}
\item Definition des Ordnungspolynoms
\item Beispiel zur Berechnung des Ordnungspolynoms
\item Ordnungspolynom in einem Faktorraum $V/U$
\item Algorithmus Ordpoly\foreignlanguage{german}{\vspace*{\fill}}
\end{itemize}

\lyxnewslide{}

\selectlanguage{german}
\textcolor{blue}{\huge Definition des Ordnungspolynoms}\\
{\huge \par}

\begin{Def}[Ordnungspolynom]
 

Sei $v\in V$. Das Polynom $o\in K[X]$ hei�t Ordnungspolynom von
$v$, wenn gilt: \\
$o$ ist normiert und $(o)=K[X]o=O(v)$ \\*[16pt]

\end{Def}

\begin{Bem} 

1. Das Ordnungspolynom ist eindeutig bestimmt, weil $K[X]$ ein   H.I.R.
ist und weil es normiert ist.

2. Allgemein gilt: Jedes Ideal $I\subset K[X]$ wird von allen Polynomen
des kleinsten Grades in $I$ erzeugt.

\end{Bem}

\vspace*{\fill}


\lyxnewslide{}

\textcolor{blue}{\huge Beispiel zur Berechnung des Ordnungspolynoms}\\
{\huge \par}

\begin{Bei}

Hier ist $K=\mathbb{Z}_{5}\;,\; V=\left(\mathbb{Z}_{5}\right)^{3}$
und $End(V)=Mat(3\times3,\mathbb{Z}_{5})$.\\
Sei $\tau=A\in Mat(3\times3,\mathbb{Z}_{5})\;,\; A:=\left(\begin{array}{ccc}
3 & 0 & 0\\
2 & 1 & 0\\
4 & 1 & 2\end{array}\right)$und $v\in\left(\mathbb{Z}_{5}\right)^{3}\;,\; v:=\left(\begin{array}{c}
1\\
2\\
1\end{array}\right)$.\\
Berechne: $Av=\left(\begin{array}{c}
3\\
4\\
3\end{array}\right)\;,\; A^{2}v=\left(\begin{array}{c}
4\\
0\\
2\end{array}\right)\;,\; A^{3}v=\left(\begin{array}{c}
2\\
3\\
0\end{array}\right)$.\\
Bestimmes $s$ minimal, so da� $v,Av,\ldots,A^{s}v$ linear abh�ngig
sind. Hier $s=3$.\\
$A^{3}v=v+4Av+A^{2}v\quad\Longrightarrow\quad A^{3}v+4A^{2}v+Av+4v=0$
\\
$\Longrightarrow\;\left(A^{3}+4A^{2}+A+4E\right)v=0\quad\Longrightarrow\quad(X^{3}+4X^{2}+X+4)v=0$.\\
Dann ist $o:=X^{3}+4X^{2}+X+4$ das Ordnungspolynom von $v$.

\end{Bei}

\vspace*{\fill}


\lyxnewslide{}

\textcolor{blue}{\huge Ordnungspolynom in $V/U$}\\
{\huge \par}

Dies erfordert den �bergang zum $K[X]$-Faktormodul $V_{\tau}/U_{\tau}$:
\\*

\begin{enumerate}
\item Die Untermoduln von $V_{\tau}$ sind gerade diejenigen Unterr�ume
$U$ von $V$ die $\tau(U)\subset U$ ( d.h. $U$ ist $\tau$-invariant
) erf�llen , hier bezeichnet mit $U_{\tau}$. 
\item Der Endomorphismus $\tau$ muss nun ver�ndert werden: Setze \[
\overline{\tau}\::\: V_{\tau}/U_{\tau}\longrightarrow V_{\tau}/U_{\tau}\quad;\:\overline{\tau}(\overline{v})=\overline{\tau(v)}\]
 Dies ist die kanonische Definition. 
\item Die Addition ist gegeben durch $\overline{v}+\overline{w}=\overline{v+w}\quad\forall\, v,w\in V_{\tau}$.
Und die Multiplikation ist gegeben durch $p\cdot\overline{v}=p(\overline{\tau})(\overline{v})=\overline{p(\tau)(v)}\quad\forall p\in K[X]\;\forall v\in V$.
$\qquad\Rightarrow$ Man rechnet ganz in $V_{\tau}$ und macht erst
zum Schluss der Rechnung den �bergang modulo $U_{\tau}$. \vspace*{\fill}
\end{enumerate}

\lyxnewslide{}

\textcolor{blue}{\huge Algorithmus Ordpoly}\\
{\huge \par}

Sei $\overline{v}\in V_{\tau}/U_{\tau}$ , und $U\:\tau$-invariant.
Weiter sei $b_{0},\ldots\,\, b_{k}$ eine Basis von $U_{\tau}$.\\
 Setzte $i:=0$\\
 Wiederhole solange die Vektoren $b_{0},\ldots,b_{k},v,\tau(v),\ldots,\tau^{i}(v)$
linear unabh�ngig sind ( dies wird mit der Funktion gauss getestet
) : setzte $i:=i+1$ . \\
 Setze $m:=i$.\\
 Die Funktion gauss liefert dann einen Vektor $f$, so da� gilt :
\[
\tau^{m}(v)=f_{0}b_{0}+\cdots+f_{k}b_{k}+f_{k+1}v+\cdots+f_{k+m}\tau^{m-1}(v)\]
\\
 Setze $w:=(f_{0},\ldots,f_{k})$ \hspace{12pt} ( Anteil in $U_{\tau}$
) \\
 Setze $f:=(f_{k+1},\ldots,f_{k+m})$ \hspace{12pt} ( Anteil im direkten
Komplement von $U_{\tau}$ ). \\
 Dann gilt: $\tau^{m}(v)-f_{m-1}\tau^{m-1}(v)-\cdots-f_{0}v=w_{0}b_{0}+\cdots+w_{k}b_{k}$.\\
 modulo $U_{\tau}$: $\quad\overline{\tau}^{m}(\overline{v})-f_{m-1}\overline{\tau}^{m-1}(\overline{v})-\cdots-f_{0}\overline{v}=\overline{0}$\\
 Setze $o:=X^{m}-f_{m-1}X^{m-1}-\cdots-f_{1}X-f_{0}$ .\\
 Dann ist $o$ das Ordnungspolynom.

\vspace*{\fill}


\lyxnewslide{}

\textcolor{blue}{\Huge 3. Der Algorithmus Minpoly}{\Huge \par}

\begin{itemize}
\item Definition des Minimalpolynoms
\item Beispiel zur Berechnung eines Minimalpolynoms
\item Theoretische Bestimmung des Minimalpolynoms
\item Bestimmung eines maximalen Vektors
\item Algorithmus Minpoly\vspace*{\fill}
\end{itemize}

\lyxnewslide{}

\textcolor{blue}{\huge Definition des Minimalpolynoms}\\
{\huge \par}

\begin{Def}[Minimalpolynom]




Sei $\tau\in End(V)$. Das Polynom $m\in K[X]$ hei�t Minimalpolynom
von $\tau$ , wenn gilt: $m$ ist normiert und $(m)=K[X]m=A(V_{\tau})=\{ p\in K[X]\mid p\cdot v=0\quad\forall v\in V_{\tau}\}$

\end{Def}

\vspace*{\fill}


\lyxnewslide{}

\textcolor{blue}{\huge Beispiel zur Berechnung eines Minimalpolynoms}\\
{\huge \par}

\begin{Bei}

Hier ist $K=\mathbb{Z}_{5}\;,\; V=\left(\mathbb{Z}_{5}\right)^{4}$
und $End(V)=Mat(4\times4,\mathbb{Z}_{5})$. 

Gegeben: Basis von $V$ : $v_{0}=\left(\begin{array}{c}
0\\
1\\
0\\
0\end{array}\right)\;,\; v_{1}=\left(\begin{array}{c}
2\\
2\\
4\\
1\end{array}\right)\;,\; v_{2}=\left(\begin{array}{c}
3\\
3\\
2\\
0\end{array}\right)\;,\; v_{3}=\left(\begin{array}{c}
2\\
0\\
0\\
0\end{array}\right)$.\\
ein Endomorphismus $A=\tau\in Mat(4\times4,\mathbb{Z}_{5})$ $\;,\; A:=\left(\begin{array}{cccc}
3 & 4 & 2 & 4\\
0 & 1 & 3 & 0\\
0 & 0 & 1 & 4\\
0 & 0 & 0 & 2\end{array}\right)$\\
und die Ordnungspolynome zu den Basisvektoren:\\
$\begin{array}{ll}
o_{0}=X^{2}+X+3=(X+2)(X+4) & \quad o_{1}=X^{2}+1=(X+2)(X+3)\\
o_{2}=X^{3}+2X+2=(X+2)(X+4)^{2} & \quad o_{3}=X+2\end{array}$

\end{Bei}

\vspace*{\fill}


\lyxnewslide{}

\textcolor{blue}{\huge Beispiel (Fortsetzung)}{\huge \par}

\enlargethispage{1cm}

\textbf{\textcolor{black}{Schritt 0}}\textcolor{black}{:} Setze: $m=o_{0}$
und $v=v_{0}$.

\textbf{\textcolor{black}{Schritt 1:}} \textcolor{black}{Setze: $\: c:=m=(X+2)(X+4)=X^{2}+X+3$}\\
\phantom{{\bf Schritt 1:} Setze: }$d:=o_{1}=(X+2)(X+3)=X^{2}+1$\textcolor{black}{}\\
\textcolor{black}{Berechne:}\begin{eqnarray*}
t & := & \ggT(c,d)=X+2\\
C & := & \R(c,\frac{d}{t})=\R(c\:,\:(X+3)\,)=(X+2)(X+4)=X^{2}+X+3\\
D & := & \R(d,\frac{c}{t})=\R(d\:,\:(X+4)\,)=(X+2)(X+3)=X^{2}+1\\
T & := & \ggT(C,D)=X+2\\
D_{2} & := & \frac{D}{T}=X+3\\
m & := & CD_{2}=(X+2)(X+3)(X+4)=X^{3}+4X^{2}+X+4\\
v & := & \frac{c}{C}\cdot v+\frac{d}{D_{2}}\cdot v_{1}=1\cdot\left(\begin{array}{c}
0\\
1\\
0\\
0\end{array}\right)+(X+2)\left(\begin{array}{c}
2\\
2\\
4\\
1\end{array}\right)=\left(\begin{array}{c}
0\\
4\\
1\\
4\end{array}\right)\end{eqnarray*}


\vspace*{\fill}


\lyxnewslide{}

\textcolor{blue}{\huge Beispiel (Fortsetzung)}{\huge \par}

\textbf{\textcolor{black}{Schritt 2:}} \textcolor{black}{Setze: $c:=m=(X+2)(X+3)(X+4)=X^{3}+4X^{2}+X+4$}\\
\phantom{{\bf Schritt 2:} Setze: }$d:=o_{2}=(X+2)(X+4)^{2}=X^{3}+2X+2$\textcolor{black}{}\\
\textcolor{black}{Berechne:}\begin{eqnarray*}
t & := & \ggT(c,d)=(X+2)(X+4)=X^{2}+X+3\\
C & := & \R(c,\frac{d}{t})=\R(\, c,\:(X+4)\,)=(X+2)(X+3)=X^{2}+1\\
D & := & \R(d,\frac{c}{t})=\R(\, d,\:(X+3)\,)=(X+2)(X+4)^{2}=X^{3}+2X+2\\
T & := & \ggT(C,D)=X+2\\
D_{2} & := & \frac{D}{T}=(X+4)^{2}=X^{2}+3X+1\\
m & := & CD_{2}=(X+2)(X+3)(X+4)^{2}=X^{4}+3X^{3}+2X^{2}+3X+1\\
v & := & \frac{c}{C}\cdot v+\frac{d}{D_{2}}\cdot v_{2}=(X+4)\left(\begin{array}{c}
0\\
4\\
1\\
4\end{array}\right)+(X+2)\left(\begin{array}{c}
3\\
3\\
2\\
0\end{array}\right)=\left(\begin{array}{c}
0\\
3\\
2\\
4\end{array}\right)\end{eqnarray*}
Hier Abbruch der Berechnungen, denn das Minimalpolynom $m$ kann nach
der Theorie nicht mehr gr��er werden.\vspace*{\fill}


\lyxnewslide{}

\textcolor{blue}{\huge Bestimmung des Minimalpolynoms}\\
{\huge \par}

\textbf{\textcolor{black}{1. Schritt:}}

Es sei $m\in K[X]$ das Minimalpolynom von $\tau$. Dann gilt \begin{equation}
K[X]m=A(V_{\tau})=\bigcap_{v\in V}O(v)\label{eq:1}\end{equation}
 nach Definition.

\vspace*{\fill}


\lyxnewslide{}

\textcolor{blue}{\huge Bestimmung des Minimalpolynoms}\\
{\huge \par}

\textbf{\textcolor{black}{2. Schritt:}}

Sei $E=(e_{0},e_{1},\ldots,e_{k})$ ein endliches Erzeugendensystem
von $V_{\tau}$. Wegen dem Satz 1 , reicht es den Schnitt in (\ref{eq:1})
nur �ber das Erzeugendensystem $E$ zu bilden:\[
K[X]m=\bigcap_{i=0}^{k}O(e_{i})\]
\\


\begin{Sat}
Sei \(e_{0}, e_{1} , \ldots , e_{k} \) ein Erzeugendensystem von dem \(R\)-Modul \(M\). Dann gilt : \[A(M) = \bigcap_{i=0}^{k} O(e_{i}) \]
\end{Sat}

\vspace*{\fill}


\lyxnewslide{}

\textcolor{blue}{\huge Bestimmung des Minimalpolynoms}\\
{\huge \par}

\textbf{\textcolor{black}{3. Schritt:}}

Sei $o_{i}\in K[X]$ das Ordnungspolynom von $e_{i}$ f�r $i=0,\ldots,k$
d.h. $K[X]o_{i}=O(e_{i})\quad(i=0,\ldots,k)$. Dann gilt nach Satz
2 : \[
K[X]m=\bigcap_{i=0}^{k}K[X]o_{i}=K[X]\cdot\kgV(o_{0},\ldots,o_{k})\]


$\Longrightarrow\quad m\in\kgV(o_{0},\ldots,o_{k})$

\begin{Sat}
Seien \( b_{0}, \ldots , b_{m} \in R \). Dann gilt: \[ \bigcap_{i=0}^{m} R b_{i} = R \cdot  \kgV(b_{0}, \ldots , b_{m}) \]
\end{Sat}



\vspace*{\fill}


\lyxnewslide{}

\textcolor{blue}{\huge Bestimmung des Minimalpolynoms}\\
{\huge \par}

\begin{Bem}
 Seien \( r_{0},\ldots,r_{m},r,s \in R \). Dann gilt:   
\begin{eqnarray*}
\kgV(r_{0},\ldots ,r_{m}) & = & \kgV(r_{0},\kgV(r_{1},\ldots ,r_{m}) ) \\
\kgV(r,s) & = & \frac{rs}{ \ggT(r,s) }
\end{eqnarray*}

\end{Bem}

\vspace*{\fill}


\lyxnewslide{}

\textcolor{blue}{\huge Bestimmung eines maximalen Vektors}\\
{\huge \par}

\begin{Sat}
Seien \(c,d \in R\). Setzte: \( \quad t:= \ggT(c,d) \: , \: C:=r(c, \frac{d}{t} )\) und \(D:= r(d, \frac{c}{t} )\). Dann gilt:
\[ \kgV(c,d) = \kgV(C,D) \]
\end{Sat}



\begin{Bei} 
Seien \(c=2^{3}3^{2}5^{4},\, d=2^{3}37^{2}\in \mathbb{Z}\). 
Dann ist: \(t=\ggT (c,d)=2^{3}3\; \Rightarrow \; \frac{c}{t}=35^{4}\: ,\: \frac{d}{t}=7^{2}\). 
Weiter ist: \( C=r(c,\frac{d}{t})=r(2^{3}3^{2}5^{4},\, 7^{2})=2^{3}3^{2}5^{4}\quad D=r(d,\, \frac{c}{t})=r(2^{3}37^{2},\, 35^{4})=2^{3}7^{2}\)
und \(T=\ggT (C,D)=2^{3}\; \) 


\( \Rightarrow \; \kgV (C,D)=\frac{CD}{T}=2^{3}3^{2}5^{4}7^{2}\).

\end{Bei}

\vspace*{\fill}


\lyxnewslide{}

\textcolor{blue}{\huge Bestimmung eines maximalen Vektors}\\
{\huge \par}

\begin{Sat}
Seien \(v_{0}, v_{1} \in M\) Torsionselemente. Ferner sei \(O(v_{0}) = R c \) und \(O(v_{1}) = R d\). Setze:   \( \quad  C:=r(c, \frac{d}{\ggT(c,d)} ) \: , \: D:= r(d, \frac{c}{\ggT(c,d)} )\) und \( D_{2}:=\frac{D}{\ggT(C,D)}\). Setze ferner \(v:= \frac{c}{C} v_{0} + \frac{d}{D_{2}} v_{1}\). Dann gilt: \[ O(v) = O(v_{0}) \cap O(v_{1}) = Rc \cap Rd = R \cdot \kgV(c,d) \]
\end{Sat}

\vspace*{\fill}


\lyxnewslide{}

\textcolor{blue}{\huge Algorithmus Minpoly}\\
{\huge \par}

Sei $v_{0},v_{1},\ldots,v_{n-1}$ eine Basis von $V$. Dann ist $\overline{v_{0}},\overline{v_{1}},\ldots,\overline{v_{n-1}}$
ein Erzeugendensystem von $V_{\tau}/U_{\tau}$ ( auch $U_{\tau}=\{0\}$
m�glich ).Bestimme Minimalpolynom von $\overline{\tau}$ .\\


\textbf{Vorarbeit:}~ Berechne Ordnungspolynom von $\overline{v_{i}}$
und speichere es in $ordpol[i]$ f�r $\: i=0,\ldots,n-1$ . 

\textbf{Schritt~0:}~ Setze $\overline{v}$ und $m$ wie folgt: \begin{eqnarray*}
\overline{v} & := & \overline{v_{0}}\\
m & := & ordpol[0]\end{eqnarray*}


\vspace*{\fill}


\lyxnewslide{}

\textcolor{blue}{\huge Algorithmus Minpoly}{\huge \par}

\textbf{Schritt~i:}~ ( F�r $i=1,\ldots,n-1$ ) \\
 Setze: $\qquad c:=m\:$ und $d:=ordpol[i]$ \\
,~wobei $m=\kgV(ordpol[0],\ldots,ordpol[i-1])$\\
 Berechne Hilfsvariblen: $\qquad t:=\ggT(c,d)\:,\: C:=r(c,\frac{d}{t})\:,\: D:=r(d,\frac{c}{t})\:,\: T:=\ggT(C,D)\:$
und $\: D_{2}:=\frac{D}{T}$. \\
 Berechne neues $m$ und neues $\overline{v}$ : \begin{eqnarray*}
m:= & C\cdot D_{2} & \quad(\textrm{ d}.\textrm{h}.\: m:=\kgV(m,orpol[i])\:\:)\\
\overline{v}:= & \frac{c}{C}\overline{v}+\frac{d}{D_{2}}\overline{v_{i}} & \quad(\textrm{ d}.\textrm{h}.\: O(\overline{v})=K[X]m\:\:)\end{eqnarray*}
 Falls $grad(m)=dimV-dimU$ verlasse Schleife vorzeitig. 

\textbf{Nacharbeit:} Normiere $m$. 

\vspace*{\fill}


\lyxnewslide{}

\textcolor{blue}{\huge Algorithmus Minpoly}{\huge \par}

\textbf{Ergebnis:}~ ~

Es ist $m=\kgV(\, ordpol[0],\ldots,ordpol[n-1]\,)$ Also ist $K[X]m=\bigcap_{i=0}^{n-1}K[X]ordpol[i]=\bigcap_{i=0}^{n}O(\overline{v_{i}})=A(V_{\tau}/U_{\tau})\:$
nach den S�tzen 1 und 2 . Damit ist $m$ das Minimalpolynom von $\overline{\tau}$
nach der Definition 1. 

Es ist $O(\overline{v})=\bigcap_{i=0}^{n-1}{O(\overline{v_{i}})}=K[X]m$
nach Satz 4. Also hat $\overline{v}$ das Polynom $m$ als Ordnungspolynom. 

\vspace*{\fill}
\end{document}
