%% LyX 1.2 created this file.  For more info, see http://www.lyx.org/.
%% Do not edit unless you really know what you are doing.
\documentclass[12pt,a4paper,german]{article}
\usepackage[T1]{fontenc}
\usepackage[latin1]{inputenc}
\pagestyle{plain}
\usepackage{amsmath}
\usepackage{amssymb}

\makeatletter

%%%%%%%%%%%%%%%%%%%%%%%%%%%%%% LyX specific LaTeX commands.
\providecommand{\LyX}{L\kern-.1667em\lower.25em\hbox{Y}\kern-.125emX\@}

%%%%%%%%%%%%%%%%%%%%%%%%%%%%%% User specified LaTeX commands.
\usepackage{amsmath, amsfonts ,amssymb, german, a4, theorem} 
\theorembodyfont{\normalfont}
\theoremstyle{break}
\newtheorem{Def}{\underline{Definition}}

\theorembodyfont{\normalfont}
\theoremstyle{break}
\newtheorem{Sat}{\underline{Satz}}

\theorembodyfont{\normalfont}
\theoremstyle{break}
\newtheorem{Bem}{\underline{Bemerkung}}

\theorembodyfont{\normalfont}
\theoremstyle{break}
\newtheorem{Bei}{\underline{Beispiel}}

\newcommand{\ggT}{ \textrm{ggT} }
\newcommand{\kgV}{ \textrm{kgV} }
\newcommand{\R}{\textrm{r} }

\usepackage{babel}
\makeatother
\begin{document}
\begin{flushleft}\textbf{\large Seminar: Computeralgebra 1}\\
\end{flushleft}

\begin{flushleft}Thema 7 : Algorithmus Minpoly und Ordpoly\\
 Hannes Buchholzer \\*[26pt]\end{flushleft}

\noindent \underbar{Generalvoraussetzungen:}\\
 Seien $K$ ein K�rper, $R$ ein euklidischer Ring, $V$ ein $K$-Vektorraum
mit $dimV=n\: (n\in \mathbb{N})$, $\tau \in End(V)$ , und $M$ ein
eindlich erzeugter Modul �ber $R$.


\section{Wiederholung}

\begin{enumerate}
\item Der Polynomring $K[X]$ ist ein euklidischer Ring, insbesondere ein
Hauptidealring (HIR). 
\item Das Annihilatorideal von M ist $A(M):=\{r\in R\mid rm=0\: \forall m\in M\}$. 
\item Das Ordnungsideal von $m\in M$ ist $O(m):=\{r\in R\mid r\cdot m=0\}$.
$O(m)$ ist ein Ideal. 
\item Ein Element $m\in M$ hei�t Torsionselement, wenn $O(m)\supsetneqq \{0\}$.
Ist jedes $m\in M$ Torsionselement, so hei�t $M$ Torsionsmodul. 
\end{enumerate}
\begin{Def}[ {\(K[X]\)-Modul \(V_{\tau}\)} ]
Sei \( f \in K[X]\) und \(v \in V\). Durch die Definition 
\( f \cdot v :=  (f(\tau))(v) \) wird \(V\) zu einem \(K[X]\)-Modul, bezeichnet mit \(V_{\tau}\). 
\end{Def}

\begin{Bem}
\begin{enumerate}
\item F�r Elemente \(k \in K \subset K[X]\) folgt aus diese Definition: \( k \cdot v = k \cdot \tau ^{0} (v) = 
k \star v \), wobei \( \star \) die Skalarmultiplikation im \(K\)-Vektorraum \(V\) bezeichnet. 
\item Der Modul \(V_{\tau}\) ist endlich erzeugt. 
\item Au�erdem ist \(V_{\tau}\) ein Torsionsmodul.  
\end{enumerate}
\end{Bem}


\section{Der Algorithmus Ordpoly}


\subsection{Definition des Ordnungspolynoms und Beispiel}

\begin{Def}
Sei \(v \in V \) und \(o \in K[X]\) normiert mit \( (o) = K[X] o = O(v)\). Dann hei�t \(o\) Ordnungspolynom von \(v\). 
\end{Def}

\begin{Bem} 
\begin{enumerate}

\item Das Ordnungspolynom ist eindeutig bestimmt, weil \(K[X]\) ein
   H.I.R. ist und weil es normiert ist.
\item Allgemein gilt: Jedes Ideal \(I \subset K[X]\) wird von allen
Polynomen des kleinsten Grades in \(I\) erzeugt. Diese sind alle assoziiert, d.h. sie unterscheiden sich nur durch Einheiten und es gibt darunter nur ein normiertes Polynom.
 \end{enumerate} 
\end{Bem}

\begin{Bei}

Hier sei $K=\mathbb{Z}_{5}$ und $V=\left(\mathbb{Z}_{5}\right)^{3}$.
Dann ist $v\in \left(\mathbb{Z}_{5}\right)^{3}\: ,\: End(V)=Mat(3\times 3,\mathbb{Z}_{5})$
und $\tau =A\in Mat(3\times 3,\mathbb{Z}_{5})$. Sei $v:=\left(\begin{array}{c}
 1\\
 2\\
 1\end{array}
\right)\quad \textrm{und}\quad A:=\left(\begin{array}{ccc}
 3 & 0 & 0\\
 2 & 1 & 0\\
 4 & 1 & 2\end{array}
\right)$. Dann ist : $Av=\left(\begin{array}{c}
 3\\
 4\\
 3\end{array}
\right)\: ,\: A^{2}v=\left(\begin{array}{c}
 4\\
 0\\
 2\end{array}
\right)$und $A^{3}v=\left(\begin{array}{c}
 2\\
 3\\
 0\end{array}
\right)$. Jetzt bestimme durch sukzessiven Test $s$ minimal, so da� $v,\, Av,\ldots ,A^{s}v$
linear abh�ngig sind. Hier ist $s=3$ , denn $v,\, Av,\, A^{2}v$
sind noch linear unabh�ngig. Also ist $v,\, Av,\, A^{2}v$ eine Basis
von $\left(\mathbb{Z}_{5}\right)^{3}$ und $A^{3}v$ l��t sich in
dieser Basis darstellen. Es ist: $A^{3}v=v+4Av+A^{2}v\; \Rightarrow \; A^{3}v-A^{2}v-4Av-v=0\; \Rightarrow \; (A^{3}+4A^{2}+1A+4E)\cdot v=0$.
Deswegen annuliert das Polynom $o=X^{3}+4X^{2}+X+4$ den Vektor $v$.
Dies ist gleichzeitig das Ordnungspolynom, denn der Grad ist wegen
der Minimalit�t von $s$ minimal. Au�erdem ist $o$ normiert.

\end{Bei}


\subsection{Ordnungspolynom in einem Faktorraum $V/U$ }

Dies erfordert den �bergang zum $K[X]$-Faktormodul $V_{\tau }/U_{\tau }$:
\\*

\begin{enumerate}
\item Dies Untermoduln von $V_{\tau }$ sind gerade diejenigen Unterr�ume
$U$ von $V$ die $\tau (U)\subset U$ erf�llen, hier bezeichnet mit
$U_{\tau }$. 
\item Der Endomorphismus $\tau $ muss nun ver�ndert werden: Setze \[
\overline{\tau }\: :\: V_{\tau }/U_{\tau }\longrightarrow V_{\tau }/U_{\tau }\quad ;\: \overline{\tau }(\overline{v})=\overline{\tau (v)}\]
 Dies ist die kanonische Definition. 
\item Die Addition ist gegeben durch $\overline{v}+\overline{w}=\overline{v+w}\quad \forall \, v,w\in V_{\tau }$.
Und die Multiplikation ist gegeben durch $p\cdot \overline{v}=p(\overline{\tau })(\overline{v})=\overline{p(\tau )(v)}\quad \forall p\in K[X]\; \forall v\in V$.
$\qquad \Rightarrow $ Man rechnet ganz in $V_{\tau }$ und macht
erst zum Schluss der Rechnung den �bergang modulo $U_{\tau }$. (
Ganz genauso wie man in $\mathbb{Z}_{5}$ rechnet ). 
\end{enumerate}

\subsection{Algorithmus Ordpoly}

Sei $\overline{v}\in V_{\tau }/U_{\tau }$ , und $U\: \tau $-invariant.
Weiter sei $b_{0},\ldots \, \, b_{k}$ eine Basis von $U_{\tau }$.\\
 Setzte $i:=0$\\
 Wiederhole solange die Vektoren $b_{0},\ldots ,b_{k},v,\tau (v),\ldots ,\tau ^{i}(v)$
linear unabh�ngig sind ( dies wird mit der Funktion gauss getestet
) : setzte $i:=i+1$ . \\
 Setze $m:=i$.\\
 Die Funktion gauss liefert dann einen Vektor $f$, so da� gilt :
\[
\tau ^{m}(v)=f_{0}b_{0}+\cdots +f_{k}b_{k}+f_{k+1}v+\cdots +f_{k+m}\tau ^{m-1}(v)\]
\\
 Setze $w:=(f_{0},\ldots ,f_{k})$ \hspace{12pt} ( Anteil in $U_{\tau }$
) \\
 Setze $f:=(f_{k+1},\ldots ,f_{k+m})$ \hspace{12pt} ( Anteil im
direkten Komplement von $U_{\tau }$ ). \\
 Dann gilt: $\tau ^{m}(v)-f_{m-1}\tau ^{m-1}(v)-\cdots -f_{0}v=w_{0}b_{0}+\cdots +w_{k}b_{k}$.\\
 modulo $U_{\tau }$: $\quad \overline{\tau }^{m}(\overline{v})-f_{m-1}\overline{\tau }^{m-1}(\overline{v})-\cdots -f_{0}\overline{v}=\overline{0}$\\
 Setze $o:=X^{m}-f_{m-1}X^{m-1}-\cdots -f_{1}X-f_{0}$ .\\
 Dies ist dann das Ordnungspolynom, denn es hat den kleinstm�glichen
Grad und ist normiert.


\section{Der Algorithmus Minpoly}


\subsection{Definition des Minimalpolynoms und Beispiel}

\begin{Def}[Minimalpolynom]
Das normierte Polynom \(m \in K[X]\) f�r das gilt: \(K[X]m = A(V_{\tau }) = \{ p \in K[X] \mid p \cdot v = 0 \quad v \in V_{\tau }\} \: \) hei\ss t Minimalpolynom von \(\tau \in End(v) \).
\end{Def}

\begin{Bei}[f�r den Algorithmus Minpoly]

Gegeben sei eine Basis von dem $\mathbb{Z}_{5}$-Vekorraum $V=\left(\mathbb{Z}_{5}\right)^{4}$:
$\; v_{0}=\left(\begin{array}{c}
 0\\
 1\\
 0\\
 0\end{array}
\right),\: v_{1}=\left(\begin{array}{c}
 2\\
 2\\
 4\\
 1\end{array}
\right),\: v_{2}=\left(\begin{array}{c}
 3\\
 3\\
 2\\
 0\end{array}
\right),\: v_{3}=\left(\begin{array}{c}
 2\\
 0\\
 0\\
 0\end{array}
\right)$~,, ein Endomorphismus von $A=\tau \in End(V)$ : $\; A=\left(\begin{array}{cccc}
 3 & 4 & 2 & 4\\
 0 & 1 & 3 & 0\\
 0 & 0 & 1 & 4\\
 0 & 0 & 0 & 2\end{array}
\right)\; $und schlie�lich noch die Ordnungspolynome zu den Basisvektoren: 

$\begin{array}{ll}
 o_{0}=X^{3}+X+3=(X+2)(X+4) & \quad o_{1}=X^{2}+1=(X+2)(X+3)\\
 o_{2}=X^{3}+2X+2=(x+2)(X+4)^{2} & \quad o_{3}=X+2\end{array}
$.

\begin{description}
\item [Schritt~0:]Setze: $\: m:=o_{0}\: $ und $\: v:=v_{0}$.
\item [Schritt~1:]Setze: $c:=m=(X+2)(X+4)\; ,\; d:=o_{1}=(X+2)(X+3)$\\
Berechne: \begin{eqnarray*}
t & := & \ggT (c,d)=(X+2)\\
C & := & \R (c,\frac{d}{t})=\R (\, (X+2)(X+4),\: (X+3)\, )=(X+2)(X+4)\\
D & := & \R (d,\frac{c}{t})=\R (\, (X+2)(X+3),\: (X+4)\, )=(X+2)(X+3)\\
T & := & \ggT (C,D)=(X+2)\\
D_{2} & := & \frac{D}{T}=(X+3)\\
m & := & D_{2}C=(X+2)(X+3)(X+4)\\
v & := & \frac{c}{C}\cdot v+\frac{d}{D_{2}}\cdot v_{1}=1\cdot \left(\begin{array}{c}
 0\\
 1\\
 0\\
 0\end{array}
\right)+(X+2)\left(\begin{array}{c}
 2\\
 2\\
 4\\
 1\end{array}
\right)=\left(\begin{array}{c}
 0\\
 4\\
 1\\
 4\end{array}
\right)
\end{eqnarray*}

\item [Schritt~2:]Setze: $c:=m=(X+2)(X+3)(X+4)\; ,\; d:=o_{2}=(X+2)(X+4)^{2}$\\
Berechne: \begin{eqnarray*}
t & := & \ggT (c,d)=(X+2)(X+4)\\
C & := & \R (c,\frac{d}{t})=R(\, (X+2)(X+3)(X+4),\: (X+4)\, )=(X+2)(X+3)\\
D & := & \R (d,\frac{c}{t})=\R (\, (X+2)(X+4)^{2},\: (X+3)\, )=(X+2)(X+4)^{2}\\
T & := & \ggT (C,D)=(X+2)\\
D_{2} & := & \frac{D}{T}=(X+4)^{2}\\
m & := & D_{2}C=(X+2)(X+3)(X+4)^{2}\\
v & := & \frac{c}{C}\cdot v+\frac{d}{D_{2}}\cdot v_{1}=(X+4)\left(\begin{array}{c}
 0\\
 4\\
 1\\
 4\end{array}
\right)+(X+2)\left(\begin{array}{c}
 3\\
 3\\
 2\\
 0\end{array}
\right)=\left(\begin{array}{c}
 0\\
 3\\
 2\\
 4\end{array}
\right)
\end{eqnarray*}
\\
Hier Abbruch der Berechnungen, denn das Minimalpolynom $\, m\, $kann
nach der Theorie nicht mehr gr��er werden.
\end{description}
\end{Bei}


\subsection{Bestimmung des Minimalpolynoms}

\textbf{\underbar{Ziel:}}Bestimmung des Minimalpolynoms von $\tau $
aus einer m�glichst geringen Anzahl von Ordnungspolynomen.


\subsubsection*{1. Schritt:}

Es sei $m\in K[X]$ das Minimalpolynom von $\tau $. Dann gilt \begin{equation}
K[X]m=A(V_{\tau })=\bigcap _{v\in V}O(v)\label{eq:1}\end{equation}
 nach Definition.


\subsubsection*{2. Schritt:}

Sei $E=(e_{0},e_{1},\ldots ,e_{k})$ ein endliches Erzeugendensystem
von $M$. Wegen dem Satz 1, reicht es den Schnitt in (\ref{eq:1})
nur �ber das Erzeugendensystem $E$ zu bilden:\[
K[X]m=\bigcap _{i=0}^{k}O(e_{i})\]
\\


\begin{Sat}
Sei \(e_{0}, e_{1} , \ldots , e_{k} \) ein Erzeugendensystem von dem \(R\)-Modul \(M\). Dann gilt : \[A(M) = \bigcap_{i=0}^{k} O(e_{i}) \]
\end{Sat}


\subsubsection*{3. Schritt:}

Sei $o_{i}\in K[X]$ das Ordnungspolynom von $e_{i}$ f�r $i=0,\ldots ,k$
d.h. $K[X]o_{i}=O(e_{i})\quad (i=0,\ldots ,k)$. Dann gilt nach einem
Resultat aus der Algebra: \[
K[X]m=K[X]s\quad \forall s\in \kgV(o_{0},\ldots ,o_{k})\]
\\


\begin{Sat}
Seien \( b_{0}, \ldots , b_{m} \in R \). Dann gilt: \[ \bigcap_{i=0}^{m} R b_{i} = R v \qquad \forall \, v \in  \kgV(b_{0}, \ldots , b_{m}) \]
\end{Sat}

\begin{Bem}
 Seien \( r_{0},\ldots,r_{m},r,s \in R \). Dann gilt:   
\begin{eqnarray*}
\kgV(r_{0},\ldots ,r_{m}) & = & \kgV(r_{0},\kgV(r_{1},\ldots ,r_{m}) ) \\
\kgV(r,s) & = & \frac{rs}{ \ggT(r,s) }
\end{eqnarray*}

\end{Bem}


\subsection{Bestimmung eines Vektors mit maximalem Ordnungspolynom}

\textbf{\underbar{Ziel:}} Bestimmung eines Vektors $v\in V$ der das
Minimalpolynom als Ordnungspolynom hat.

\begin{Sat}
Seien \(c,d \in R\). Setzte: \( \quad t:= \ggT(c,d) \: , \: C:=r(c, \frac{d}{t} )\) und \(D:= r(d, \frac{c}{t} )\). Dann gilt:
\[ \kgV(c,d) = \kgV(C,D) \]
\end{Sat}

\begin{Bei} Seien $c=2^{3}3^{2}5^{4},\, d=2^{3}37^{2}\in \mathbb{Z}$.
Dann ist: $t=\ggT (c,d)=2^{3}3\; \Rightarrow \; \frac{c}{t}=35^{4}\: ,\: \frac{d}{t}=7^{2}$.
Weiter ist: $C=r(c,\frac{d}{t})=r(2^{3}3^{2}5^{4},\, 7^{2})=2^{3}3^{2}5^{4}\quad D=r(d,\, \frac{c}{t})=r(2^{3}37^{2},\, 35^{4})=2^{3}7^{2}$
und $T=\ggT (C,D)=2^{3}\; \Rightarrow \; \kgV (C,D)=\frac{CD}{T}=2^{3}3^{2}5^{4}7^{2}$.

\end{Bei}

\begin{Sat}
Seien \(v_{0}, v_{1} \in M\) Torsionselemente. Ferner sei \(O(v_{0}) = R c \) und \(O(v_{1}) = R d\). Setze:   \( \quad  C:=r(c, \frac{d}{\ggT(c,d)} ) \: , \: D:= r(d, \frac{c}{\ggT(c,d)} )\) und \( D_{2}:=\frac{D}{\ggT(C,D)}\). Setze ferner \(v:= \frac{c}{C} v_{0} + \frac{d}{D_{2}} v_{1}\). Dann gilt: \[ O(v) = O(v_{0}) \cap O(v_{1}) = R a \quad \forall a \in \kgV(C,D_{2})\]
\end{Sat}


\subsection{\noindent Algorithmus Minpoly}

Sei $v_{0},v_{1},\ldots ,v_{n}$ eine Basis von $V$. Dann ist $\overline{v}_{0},\overline{v}_{1},\ldots ,\overline{v}_{n}$
eine Basis von $V_{\tau }/U_{\tau }$ ( auch $U_{\tau }=\{0\}$ m�glich
). \\


\begin{description}
\item [Vorarbeit:~]Berechne Ordnungspolynom von $v_{i}$ und speichere
es in $ordpol[i]$ f�r $\: i=0,\ldots ,n-1$ . 
\item [Schritt~0:~]Setze $v$ und $m$ wie folgt: \begin{eqnarray*}
v & := & v_{0}\\
m & := & ordpol[0]
\end{eqnarray*}

\item [Schritt~i:~]( F�r $i=1,\ldots ,n-1$ ) \\
 Setze: $\qquad c:=m\: $ und $d:=ordpol[i]\quad $ , wobei $m=\kgV(ordpol[0],\ldots ,ordpol[i-1])$\\
 Berechne Hilfsvariblen: $\qquad t:=\kgV(c,d)\: ,\: C:=r(c,\frac{d}{t})\: ,\: D:=r(d,\frac{c}{t})\: ,\: T:=\ggT(C,D)\: $
und $\: D_{2}:=\frac{D}{T}$. \\
 Berechne neues $m$ und neues $v$ : \begin{eqnarray*}
m:= & C\cdot D_{2} & \quad \textrm{( d.h.}\: m:=\kgV(m,orpol[i])\: \: )\\
v:= & \frac{c}{C}v+\frac{d}{D_{2}}v_{i} & \quad \textrm{( d.h.}\: O(v)=K[X]m\: \: )
\end{eqnarray*}
 Falls $grad(m)=n$ verlasse Schleife vorzeitig. 
\item [Nacharbeit:]Normiere $m$. 
\item [Ergebnis:~]~

\begin{enumerate}
\item Es ist $m=\kgV(\, ordpol[0],\ldots ,ordpol[n-q]\, )$ Also
ist $K[X]m=\bigcap _{i=0}^{n}{O(v_{i})}=A(V_{\tau })\: $ nach den
S�tzen 1 und 2 . Damit ist $m$ das Minimalpolynom von $\tau $ nach
der Definition 1. 
\item Es ist $O(v)=\bigcap _{i=0}^{n}{O(v_{i})}=K[X]m$ nach Satz 4. Also
hat $v$ das Polynom $m$ als Ordnungspolynom. \end{enumerate}
\end{description}

\end{document}
