\documentclass[11pt]{article}
\usepackage{amscd,amssymb,amsmath}
\usepackage[utf8]{inputenc}
\usepackage{graphicx} %extended version vom package 'graphics'
\DeclareGraphicsRule{.tif}{bmp}{jpg}{}
\usepackage{ngerman}
\usepackage{enumerate}

\pagestyle{empty}
\parindent=0pt
\setlength{\topmargin}{-1cm} \setlength{\headsep}{0cm}
\setlength{\headheight}{2.5cm} \setlength{\oddsidemargin}{0cm}
\setlength{\textwidth}{16cm} \setlength{\textheight}{24cm}

\begin{document}

\centerline{\large \bf Zusammenfassung}
\vspace{5mm}
Diese Dissertation beschäftigt sich mit dem mathematischen Teilgebiet der Gitterbasenreduktion. Aufbauend aus den Erkenntnissen der Diplomarbeit „Gitterbasenreduktion mit Random Sampling“ werden verschiedene Modifikationen am ursprünglichen LLL- bzw. BKZ-Verfahren vorgenommen: Es wird der von C. Schnorr entwickelte Ansatz, den LLL-Austauschschritt um Tiefeneinfügungen zu erweitern, aufgegriffen und eine alternative Methode zum Basisaustausch für das BKZ-Verfahren vorgestellt. Ferner werden zwei unterschiedliche Verfahren von A. Wassermann und P. Nguyen zum Abschneiden von Enumerationsbäumen beschrieben.\\
\\
Die Random Sampling Strategie von Schnorr wurde überarbeitet, um ein schlechtes GSA-Verhalten des Gitters zu berücksichtigen und eine neuartige Strategie von Buchmann und Ludwig wurde implementiert, bei der das GSA-Verhalten vollkommen irrelevant ist.
\\
\\
Schließlich wird ein grundlegendes, heuristisches Bewertungskonzept für Gittervektoren entwickelt, das im Rahmen eines von T. Vidick und P. Nguyen beschriebenen Siebverfahrens, Anwendung findet.
\\
\\
Mit Hinblick auf die Qualität der erreichten Gitter-Reduktion für schwierige Market-Split-Probleme in Dimensionen $\approx120$, liefern diese neuen Methoden hervorragende Ergebnisse in äußerst kurzer Zeit (ca. 5 Stunden auf einem 3 GHz Rechner). Auch für Problemdimensionen $>500$ sind die Resultate durchaus noch zufriedenstellend - allerdings ist hierbei der Rechenaufwand ($>7$ Tage) nicht mehr zu vernachlässigen.\\
\\
Im Vergleich mit dem kommerziellen Programm CPLEX, das einen völlig anderen Ansatz zur Lösung von ganzzahlig-linearen Gleichungssystemen verfolgt, konnten sogar sehr gute Ergebnisse erzielt werden.

\end{document}